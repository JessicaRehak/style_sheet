\documentclass[10pt, letter]{article}
\usepackage{amsmath}
\usepackage{fancyhdr}
\usepackage{geometry}
\usepackage{hyperref}
\geometry{
  letterpaper,
  left=0.5in,
  right=0.5in,
  bottom=1in,
  top=1in}
\setlength\parindent{0pt} % No indentation
\usepackage{multicol}
\renewcommand\labelitemi{}
\renewcommand\labelitemii{}
\pagestyle{fancy}
\lhead{Style Guide}
\chead{}
\rhead{J.S. Rehak}
\lfoot{}
\cfoot{\thepage}
\rfoot{{\small{}Based on AIP Style Guide (4th Edition)}}
\begin{document}

\tableofcontents
\pagebreak
\section{General Guidance}\label{sec:general_guidance}

\begin{multicols}{2}
Tables, Figures and Captions
\begin{itemize}
\item In Eq. (13), (14), and (16)
\item In Fig. 4(a)
\item In Ref. 5
\end{itemize}

Dates and numbers
\begin{itemize}
\item 02 February 2016 (no commas) 
\item four or fewer numbers closed up:
  \begin{itemize}
  \item 1200
  \item 24.0032 cm
  \end{itemize}
\item Five or more digits, spaces instead of commas:
  \begin{itemize}
  \item 12 000
  \item 24.077 89 cm
  \end{itemize}
\item one throgh ten
\item 11,12 and above
\item 2x2 matrix (numerals)
\item 0.03 and 106.0(no ``naked'' decimal points
\item 6 V (number before units are always numerals)
\item 1D, 2D, 3D
\end{itemize}

Punctuation
\begin{itemize}
\item en-dash: Paris--London train, (1950--), University of Wisconsin--Madison
\item serial commas (a, b, and c)
\item hyphenate multi-word modifiers: macro-time
\item parenthesis:
  \begin{itemize}
  \item inserted into another sentence, no period (such as this).
  \item isolated, period inside. (Such as this.)
  \item pairs surrounded letters in innumerated list (a) and (b)
  \end{itemize}
\item possessives: Smith and Green's theory
\item plurals:
  \begin{itemize}
  \item 1950s
  \item x's, K's
  \end{itemize}
\item quotation marks after commas and periods, before colons and
  semi-colons
\end{itemize}

Abbreviations
\begin{itemize}
\item Plural add 's: LCAO's
\end{itemize}

\pagebreak

\section{Specific words and terms}\label{sec:words}

\subsubsection*{A}
\begin{itemize}
\item $\alpha$ particle
\item \textit{ad hoc}
\item \textit{\`{a} la}
\item anti-compounds closed (antilogarithm)
\end{itemize}

\subsubsection*{B}
\begin{itemize}
\item burnup (n)
\end{itemize}


\subsubsection*{C}
\begin{itemize}
\item Cartesian
\item collision-flux estimator
\item cross-section (n)
\item cross term
\end{itemize}

\subsubsection*{D}
\begin{itemize}
\item delta-tracking
\item Doppler
\item downscatter
\end{itemize}


\subsubsection*{E}
\begin{itemize}
\item eigenfunction
\item eigenvalue
\end{itemize}
\subsubsection*{F}
\begin{itemize}
\item Fourier transform/analysis/spectra
\end{itemize}
\subsubsection*{G}
\begin{itemize}
\item Gauss-Seidel (adj)
\end{itemize}
\subsubsection*{H}
\begin{itemize}
\item half-compound hyphenated:
  \begin{itemize}
  \item half-life
  \end{itemize}
\item halfway
\end{itemize}

\subsubsection*{I}
\begin{itemize}
\item indexes (to book)
\item indices (to variable)
\item \textit{in situ}
\end{itemize}

\subsubsection*{J}
\subsubsection*{K}

\subsubsection*{L}
\begin{itemize}
\item Laplacian
\item l.h.s.
\item lifetime
\end{itemize}

\subsubsection*{M}
\begin{itemize}
\item Maxwell(ian)
\item midpoint
\item modeling
\item multigroup
\item multivarient
\end{itemize}

\subsubsection*{N}
\begin{itemize}
\item non-compound closed:
  \begin{itemize}
  \item nonelastic
  \item nonradioactive
  \item \textit{but} proper noun, symbol, numeral:
  \item non-Fermi
  \item 12-fold
  \end{itemize}
\end{itemize}

\subsubsection*{O}
\subsubsection*{P}
\begin{itemize}
\item path length
\end{itemize}
\subsubsection*{Q}
\subsubsection*{R}
\begin{itemize}
\item radioactive
\item ray tracing
\item r.h.s.
\item runtime
\end{itemize}

\subsubsection*{S}
\begin{itemize}
\item setup
\item self-compound hyphenated:
  \begin{itemize}
  \item self-shielded (adj)
  \end{itemize}
\item semiempirical
\item semi-infinite
\end{itemize}

\subsubsection*{T}
\begin{itemize}
\item track length
\item track-length estimator
\end{itemize}
\subsubsection*{U}
\begin{itemize}
\item upscatter
\item uranium
\end{itemize}
\subsubsection*{V}
\subsubsection*{W}
\begin{itemize}
\item waveheight
\item wavelength
\end{itemize}

\subsubsection*{X}
\begin{itemize}
\item x ray (n) 
\item x-ray (adj)
\end{itemize}
\subsubsection*{Y}
\subsubsection*{Z}

\pagebreak

\section{Math and notation}\label{sec:math}

\subsubsection*{Cross-sections}
\begin{itemize}
\item macroscopic: $\tilde{\sigma}$
\item microscopic: $\sigma$
\end{itemize}

\subsubsection*{Matrices}
\begin{itemize}
\item Bold capital letters, $\mathbf{A}$.
\item Use brackets (\verb|bmatrix|) for normal matrix, pipes
  ($\verb|vmatrix|$) for
  determinants, and double pipes ($\verb|Vmatrix|$) for a matrix norm.
\end{itemize}

\subsubsection*{Vectors}
\begin{itemize}
\item Topped with an arrow, $\vec{\phi}$. Vector superscripts must be shifted
slightly using \verb|\vec{\phi}^{\,\ell}|. For comparison:
\begin{center}
  \begin{tabular}[h!]{lcl}
    \verb|\vec{\phi}^{\ell}| & : & $\vec{x}^{\ell}$ \\
    \verb|\vec{x}^{\,\ell}| & : & $\vec{x}^{\,\ell}$
  \end{tabular}
\end{center}
\item Use hats to denote unit vectors, $\hat{\Omega}$.
\item In general, if a vector is made up of other vectors, use a
  capital letter for the larger vector, and lowercase for the smaller
  vectors.
\begin{equation*}
    \vec{\Phi} =
    \begin{bmatrix}
      \vec{\phi_0} \\ \vec{\phi_1}
    \end{bmatrix}
\end{equation*}
\end{itemize}









\end{multicols}
\end{document}
%%% Local Variables:
%%% mode: latex
%%% TeX-master: t
%%% End:
