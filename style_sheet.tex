\documentclass[10pt, letter]{article}
\usepackage{amsmath}
\usepackage{fancyhdr}
\usepackage{geometry}
\usepackage[hidelinks]{hyperref}
\usepackage{graphicx}
% Correct subfigure numbering
\usepackage{caption, subcaption}
\renewcommand\thesubfigure{(\alph{subfigure})}
\captionsetup[sub]{
  labelformat=simple
}
% ===========================
% Colored code blocks
\usepackage{xcolor, listings}
\definecolor{light-gray}{gray}{0.95}
\lstset{backgroundcolor=\color{light-gray},basicstyle=\footnotesize\ttfamily}
% ===========================
% C++ name formatting
\newcommand{\Cpp}[1][]{\textrm{C\nolinebreak[4]\hspace{-.05em}\raisebox{.4ex}{\tiny\bf
      ++}#1}}
% ===========================
\geometry{
  letterpaper,
  left=0.5in,
  right=0.5in,
  bottom=1in,
  top=1in}
\setlength\parindent{0pt} % No indentation
\usepackage{multicol}
\renewcommand\labelitemi{}
\renewcommand\labelitemii{}
\pagestyle{fancy}
\lhead{Style Guide}
\chead{}
\rhead{J.S. Rehak}
\lfoot{}
\cfoot{\thepage}
\rfoot{{\small{}Based on AIP Style Guide (4th Edition)}}
\begin{document}
\raggedbottom
\tableofcontents
\pagebreak
\section{General Guidance}\label{sec:general_guidance}

\begin{multicols}{2}
  \raggedcolumns

\subsection{Figure numbering}
Tables, Figures and Captions (see Sec.~\ref{sec:latex} for formatting
properly in \LaTeX).
\begin{itemize}
\item In Eq.~(13),~(14), and~(16)
\item In Fig.~4(a)
\item In Ref.~5
\end{itemize}
Place the caption under figures and images and above tables.

\subsection{Dates and numbers}
\label{sec:dates_and_numbers}
\begin{itemize}
\item 02 February 2016 (no commas) 
\item four or fewer numbers closed up:
  \begin{itemize}
  \item 1200
  \item 24.0032 cm
  \end{itemize}
\item Five or more digits, spaces instead of commas:
  \begin{itemize}
  \item 12 000
  \item 24.077 89 cm
  \end{itemize}
\item one throgh ten
\item 11,12 and above
\item 2x2 matrix (numerals)
\item 0.03 and 106.0 (no ``naked'' decimal points
\item 6 V (number before units are always numerals)
\item 1D, 2D, 3D
\end{itemize}

\subsection{Punctuation}
\label{sec:punctuation}
\begin{itemize}
\item en-dash: Paris--London train, (1950--), University of Wisconsin--Madison
\item serial commas (a, b, and c)
\item hyphenate multi-word modifiers: macro-time
\item parenthesis:
  \begin{itemize}
  \item inserted into another sentence, no period (such as this).
  \item isolated, period inside. (Such as this.)
  \item pairs surrounded letters in innumerated list (a) and (b)
  \end{itemize}
\item possessives: Smith and Green's theory
\item plurals:
  \begin{itemize}
  \item 1950s
  \item x's, K's
  \end{itemize}
\item quotation marks after commas and periods, before colons and
  semi-colons
\item in general, place ``e.g.'' and ``i.e.'' in parenthesis, not
  commas and include a comma after (e.g., like this). 
\end{itemize}

\subsection{Abbreviations}
\label{sec:abbreviations}
\begin{itemize}
\item Plural add 's: LCAO's
\end{itemize}

\pagebreak

\section{Specific words and terms}\label{sec:words}

\subsubsection*{A}
\begin{itemize}
\item $\alpha$ particle
\item \textit{ad hoc}
\item \textit{\`{a} la}
\item anti-compounds closed (antilogarithm)
\end{itemize}

\subsubsection*{B}
\begin{itemize}
\item burnup (n)
\end{itemize}


\subsubsection*{C}
\begin{itemize}
\item Cartesian
\item collision-flux estimator
\item cross-section (n)
\item cross term
\end{itemize}

\subsubsection*{D}
\begin{itemize}
\item delta-tracking
\item Doppler
\item downscatter
\end{itemize}


\subsubsection*{E}
\begin{itemize}
\item eigenfunction
\item eigenvalue
\end{itemize}
\subsubsection*{F}
\begin{itemize}
\item Fourier transform/analysis/spectra
\end{itemize}
\subsubsection*{G}
\begin{itemize}
\item Gauss-Seidel (adj)
\end{itemize}
\subsubsection*{H}
\begin{itemize}
\item half-compound hyphenated:
  \begin{itemize}
  \item half-life
  \end{itemize}
\item halfway
\end{itemize}

\subsubsection*{I}
\begin{itemize}
\item indexes (to book)
\item indices (to variable)
\item \textit{in situ}
\end{itemize}

\subsubsection*{J}
\subsubsection*{K}

\subsubsection*{L}
\begin{itemize}
\item Laplacian
\item l.h.s.
\item lifetime
\end{itemize}

\subsubsection*{M}
\begin{itemize}
\item Maxwell(ian)
\item midpoint
\item modeling
\item multigroup
\item multivarient
\end{itemize}

\subsubsection*{N}
\begin{itemize}
\item non-compound closed:
  \begin{itemize}
  \item nonelastic
  \item nonradioactive
  \item \textit{but} proper noun, symbol, numeral:
  \item non-Fermi
  \item 12-fold
  \end{itemize}
\end{itemize}

\subsubsection*{O}
\subsubsection*{P}
\begin{itemize}
\item path length
\end{itemize}
\subsubsection*{Q}
\subsubsection*{R}
\begin{itemize}
\item radioactive
\item ray tracing
\item r.h.s.
\item runtime
\end{itemize}

\subsubsection*{S}
\begin{itemize}
\item setup
\item self-compound hyphenated:
  \begin{itemize}
  \item self-shielded (adj)
  \end{itemize}
\item semiempirical
\item semi-infinite
\end{itemize}

\subsubsection*{T}
\begin{itemize}
\item track length
\item track-length estimator
\end{itemize}
\subsubsection*{U}
\begin{itemize}
\item upscatter
\item uranium
\end{itemize}
\subsubsection*{V}
\subsubsection*{W}
\begin{itemize}
\item waveheight
\item wavelength
\end{itemize}

\subsubsection*{X}
\begin{itemize}
\item x ray (n) 
\item x-ray (adj)
\end{itemize}
\subsubsection*{Y}
\subsubsection*{Z}

\pagebreak

\section{Math and notation}\label{sec:math}

\subsection{Cross-sections}
Macroscopic cross-sections are used so infrequently in neutronics that
reserving the use of capital sigma, $\Sigma$, is inefficient. Use the
following notation to differentiate between the two:
\begin{itemize}
\item macroscopic: $\tilde{\sigma}$
\item microscopic: $\sigma$
\end{itemize}

\subsection{Integrals}\label{sec:integrals}
\begin{itemize}
\item To ease reading, for single-integration integrals with terms that do not
  end with a parenthesis,  place the differential on the right side, with a space between the
  last variable and the $d$,
  \begin{lstlisting}
    \int f(x)\,dx
\end{lstlisting}
  \begin{equation*}
    \int f(x)\,dx
  \end{equation*}
  For multiple-integration integrals with terms that do not start with
  a parenthesis, place the differential immediately after its corresponding
  integral symbol, place a space after the last differential:
\begin{lstlisting}
    \int_{0}^{1}dx \int_{0}^{1} dy\,f(x,y)
\end{lstlisting}
  \begin{equation*}
    \int_{0}^{1}dx \int_{0}^{1} dy\,f(x,y)
  \end{equation*}
  The space helps emphasize that the differential is a single variable. 
\item Use parenthesis or brackets for any integral with multiple
  terms, the extra space is not required,
\begin{lstlisting}
    \int \left[f(x) + g(x)\right]dx
\end{lstlisting}
  \begin{equation*}
    \int \left[f(x) + g(x)\right]dx
  \end{equation*}
\begin{lstlisting}
    \int_{0}^{1}dx \int_{0}^{1} dy\left[f(x,y) + g(x,y)\right]
\end{lstlisting}
  \begin{equation*}
    \int_{0}^{1}dx \int_{0}^{1} dy\left[f(x,y) + g(x,y)\right]
  \end{equation*}
\end{itemize}




\subsection{Matrices}
\begin{itemize}
\item Bold capital letters, $\mathbf{A}$.
\item Use brackets (\verb|bmatrix|) for normal matrix, pipes
  ($\verb|vmatrix|$) for
  determinants, and double pipes ($\verb|Vmatrix|$) for a matrix norm.
\end{itemize}

\subsection{Vectors}
\begin{itemize}
\item Topped with an arrow, $\vec{\phi}$. Vector superscripts must be shifted
slightly using \verb|\vec{\phi}^{\,\ell}|. For comparison:
\begin{center}
  \begin{tabular}[h!]{lcl}
    \verb|\vec{\phi}^{\ell}| & : & $\vec{\phi}^{\ell}$ \\
    \verb|\vec{\phi}^{\,\ell}| & : & $\vec{\phi}^{\,\ell}$
  \end{tabular}
\end{center}
\item Use hats to denote unit vectors, $\hat{\Omega}$.
\item In general, if a vector is made up of other vectors, use a
  capital letter for the larger vector, and lowercase for the smaller
  vectors.
\begin{equation*}
    \vec{\Phi} =
    \begin{bmatrix}
      \vec{\phi_0} \\ \vec{\phi_1}
    \end{bmatrix}
\end{equation*}
\end{itemize}
\end{multicols}
\pagebreak

\section{Other \LaTeX{} specific items}\label{sec:latex}

\subsection{Figures}\label{sec:figures}
\begin{itemize}
\item Place the \verb|\label{}| for figures inside the caption to ensure
  correct references:
\begin{lstlisting}
\caption{This is the caption.\label{fig:ref}}
\end{lstlisting}
\end{itemize}

\subsection{Labels}
\label{sec:labels}
\begin{itemize}
\item Use the following formats for labels:
  \begin{center}
    \begin{tabular}{ll}
      Chapters & \verb|\label{sec:chapter_name}| \\
      Sections & \verb|\label{sec:chapter_name:section_name}|
    \end{tabular}
  \end{center}
\end{itemize}

\subsection{Package settings}
\begin{itemize}
\item Always hide boxes from hyperref package:
\begin{lstlisting}
  \usepackage[hidelinks]{hyperref}
\end{lstlisting}
\end{itemize}

\subsection{Programming language names}
\begin{itemize}
\item For the \Cpp{} programming language use:
\begin{lstlisting}
\newcommand{\Cpp}[1][]{\textrm{C\nolinebreak[4]\hspace{-.05em}\raisebox{.4ex}{\tiny\bf ++}#1}}
\end{lstlisting}
  This greatly improves the look of the name:
      \begin{center}
    \begin{tabular}[h!]{ll} 
      \verb|C++17| & C++ \\
      \verb|\Cpp{17}| & \Cpp{17}
    \end{tabular}
  \end{center}
\end{itemize}

\subsection{References and citations}

\begin{itemize}
\item For equations, use the amsmath \verb|\eqref{label}| function.
  \begin{equation}\label{eq:relativity}
    E = mc^2
  \end{equation}
  This correctly formats \verb|Eq.~\eqref{eq:relativity}| as
  Eq.~\eqref{eq:relativity}.
\item Use \verb|Sec.~\ref{sec:latex}| for sections, which correctly
  formats as Sec.~\ref{sec:latex}.
\item For figures, use \verb|Fig.~\ref{fig:image}|, which correctly
  formats as Fig.~\ref{fig:multifig}.
\item For subfigures, include the packages and commands:
\begin{lstlisting}
\usepackage{caption, subcaption}
\renewcommand\thesubfigure{(\alph{subfigure})}
\captionsetup[sub]{labelformat=simple}
\end{lstlisting}
and reference the subfigure itself,
which will format correctly as Fig.~\ref{fig:multifig_a}. See
documentation for these packages if needed.
\end{itemize}

\begin{figure}[h]
  \centering
  \begin{subfigure}[b]{0.2\textwidth}
    \includegraphics[width=\textwidth]{example-image.pdf}
    \caption{}\label{fig:multifig_a}
  \end{subfigure}\hspace{10pt}
    \begin{subfigure}[b]{0.2\textwidth}
    \includegraphics[width=\textwidth]{example-image.pdf}
    \caption{\label{fig:multifig_b}}
  \end{subfigure}
  \caption{Subfigure with parts (a) and (b).}
  \label{fig:multifig}
\end{figure}


\subsection{Spacing}
\begin{itemize}
\item For abbreviations use \verb|.\| or \verb|.~| if a tie is needed
  (titles or other words that should not be separated).
  \begin{center}
    \begin{tabular}[h!]{ll}
      Normal & e.g. this example; seen in Fig. 1 \\
      Proper & e.g.\ this example; seen in Fig.~1
    \end{tabular}
  \end{center}
  Note: the bibliography handles this correctly already.
\item Specify interspace spacing, \verb|\@.| if a capital letter ends
  a sentence:
    \begin{center}
    \begin{tabular}[h!]{ll} 
      Normal & The code is called BART. As you can see. \\
      Proper & The code is called BART\@. As you can see.
    \end{tabular}
  \end{center}
\end{itemize}  


\end{document}
%%% Local Variables:
%%% mode: latex
%%% TeX-master: t
%%% End:
